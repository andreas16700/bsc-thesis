\chapter{Synchronization System Case Study}
\etocsettocstyle{\rule{\textwidth}{1pt}}{\rule{\textwidth}{1pt}} % style for toc
\localtableofcontents
\label{chap:synchronization}

In eCommerce, synchronization systems are often needed to ensure that the online store reflects the current state of the store's products' logistics. Brick-and-mortar stores manage their inventory with logistical software system. A synchronization system that ensures the consistency and truthfulness of the online store is crucial.
The main focus of this chapter is on a practical comparison between serverless and monolithic implementations of such a synchronization system. Synthetic workloads are generated and two mock servers, simulating a logistics system and a Shopify store are used.

\section{System Overview}
Suppose a brick-and-mortar store sells products. Each product has different variants. The store keeps track of its inventory with a logistics system. That system provides an API endpoint with which one can query data and modify it.
The store wishes to have an online eCommerce store. The platform of the store has an API endpoint for adding and modifying resources, such as the products.
A system is needed that ensures that the online store reflects the current state of the physical store. That means that Every product that exists on the physical store, should exist on the online store with the correct quantity. If a product sells out on the physical store, it is the job of the system to ensure that the lack of stock is present on the online store.
\subsection{Main Components}
PS will denote the server that runs the physical store's entrypoint.
SH will denote the server that runs the online store's endpoint. 
Syncer is the object that synchronizes a product between the two endpoints.


