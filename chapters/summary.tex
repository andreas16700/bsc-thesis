\chapter*{Summary}

This thesis delves into the potential of Swift in the rapidly growing field of serverless computing. Serverless computing has gained significant attention due to its scalability and cost-effectiveness, making the choice of programming language a crucial factor in this context.

Swift, a language renowned for its popularity on Apple platforms, is known for its speed, safety, and reliability. Despite its widespread use in the Apple ecosystem, Swift's potential as a serverless language remains largely unexplored. This thesis aims to bridge this gap, driven by Swift's promise of speed and safety.

The methodology for evaluating Swift's capabilities as a serverless language involves a two-pronged approach. Initially, a qualitative comparison is conducted with popular serverless languages to investigate potential performance benefits Swift might provide. Subsequently, Swift's capabilities are evaluated through a case study, comparing a serverless to a monolithic implementation of a synchronization system.

The key findings from the research indicate that while Swift holds promise, its lack of community support and unstable Linux support with various functionalities missing, make it unsuitable for production use as a systems language, let alone serverless. Furthermore, the case study highlighted OpenWhisk's need to support intra-concurrency to fully utilize the hardware and achieve real concurrency in invoking actions.

In terms of benefits, Swift is enjoyable to write in and is a very expressive language. Most errors are caught at compile time, saving critical time and effort. Its seamless Copy-on-Write (CoW) support has the potential to greatly benefit memory and performance-critical environments, such as serverless. However, the limitations, particularly the lack of community support and ambiguity in many features in regards to their Linux support, pose significant challenges.

In conclusion, this thesis provides a comprehensive exploration of Swift as a serverless language, contributing to the ongoing discourse in this area and highlighting areas for further research.

