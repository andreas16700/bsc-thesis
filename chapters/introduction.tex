\chapter{Introduction}
\etocsettocstyle{\rule{\textwidth}{1pt}}{\rule{\textwidth}{1pt}} % style for toc
\localtableofcontents
\section{Background}

The landscape of computing has witnessed a paradigm shift in recent years, with the emergence of serverless computing. Serverless computing enables developers to focus on writing their application's code without worrying about underlying infrastructure management, provisioning, and scaling. As a result, serverless computing has gained popularity as a cost-effective and scalable alternative to traditional web and application development practices. Some of the many advantages of serverless computing include auto-scaling, elimination of idle server costs, pay-per-use pricing, and seamless scaling to handle fluctuating workloads.

In the realm of serverless computing, one's choice of a programming language is crucial. It affects not only the performance but also the ease of development and maintenance of serverless applications. The focus of this thesis is the Swift programming language, which has demonstrated its potential for speed, safety, and reliability, particularly in Apple platforms.

\section{Swift in Serverless Computing}

Swift is a statically typed, general-purpose, multi-paradigm programming language developed by Apple Inc. for iOS, macOS, watchOS, and other platforms. Since its release in 2014, Swift has gained significant popularity and achieved acclaim for its efficiency, safety, and ease of use. The Swift programming language demonstrates potential as a viable serverless language due to its inherent speed, safety, and its compatibility with OpenWhisk, which is the selected environment for this study.

\section{Choice of Serverless Environment: OpenWhisk}

OpenWhisk~\cite{openwhisk2023} is an event-driven, open-source serverless computing platform that supports a wide range of programming languages, including Swift, Python, Java, and Node.js. The reasons for selecting OpenWhisk as the environment for this study include its widespread adoption in serverless computing contexts, its flexibility, and its support for Swift as a first-class language. For the purpose of the comparative analysis in this thesis, we will examine the advantages and disadvantages of using Swift against other popular serverless computing languages such as Python, Java, and Node.js, all of which have established themselves in this domain.

\section{Case Study: Synchronization System in eCommerce}

A case study detailing a monolithic and serverless implementation of a synchronization system in the eCommerce industry will be presented in this thesis. The synchronization system serves a critical function in ensuring consistency and accuracy across online stores, as it aids in maintaining inventory, order management, and customer accounts up-to-date. This case study will help demonstrate Swift's viability as a serverless language in a real-world scenario and contribute to the broader discussion on the appropriateness of Swift for serverless computing.
\section{Objective}

The overarching objective of this thesis is to provide a comprehensive analysis of Swift as a serverless language, examining its potential for performance and ease of development within the OpenWhisk environment. By conducting an in-depth comparison with other common serverless languages and illustrating Swift's applicability through a practical case study, this thesis aims to contribute valuable insights and perspectives, further advancing the understanding of Swift's role in serverless computing.
\section{Research Questions}
With the background and context established, this thesis will attempt to answer the following research questions:

\subsection{Performance Overview}
How would the performance of Swift in serverless computing compare to other languages, such as Python, Java, and Node.js? This involves examining any potential benefits Swift could provide over other languages in a serverless context, as well as any possible performance limitations or challenges it may present.

\subsection{Ease of Development}
To what degree is Swift an accessible and efficient language for developers in a serverless environment? This question will be addressed through evaluating the development experience of Swift compared to other serverless languages, investigating aspects such as language features, syntax, tooling, and libraries that impact the ease of development of serverless functions in Swift.



%\section{Thesis Structure}
%
%The structure of this thesis is as follows:
%\begin{itemize}
%    \item Chapter 2: A comprehensive review of serverless computing, its benefits, challenges, and related programming languages.
%    \item Chapter 3: An introduction to Swift, its language features, and a discussion of its suitability for serverless computing.
%    \item Chapter 4: A detailed comparison of Swift with other popular languages for serverless computing: Python, Java, and Node.js, guided by the research questions on performance and ease of development.
%    \item Chapter 5: An overview of the OpenWhisk environment, its key components, and a justification for its selection in this thesis.
%    \item Chapter 6: A presentation of the case study comparing monolithic and serverless implementations of a synchronization system in eCommerce.
%    \item Chapter 7: An analysis of the case study, discussing the benefits and drawbacks of using Swift in this context.
%    \item Chapter 8: Conclusion and future research directions.
%\end{itemize}