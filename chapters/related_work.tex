\chapter{Related Work}
\label{chap:relatedwork}

This chapter provides a brief review of the existing literature relevant to the use of Swift as a serverless language on OpenWhisk. 

\section{Introduction}
The concept of serverless computing has gained considerable attention in recent years. The promise of serverless is to simplify cloud programming by eliminating the need for developers to manage servers and allowing them to focus solely on their application logic. Jonas et al. \cite{Jonas:EECS-2019-3} provide a thorough overview of this paradigm, tracing its history from the early days of cloud computing to the present. They argue that serverless computing is the next step in the evolution of abstraction in computing, comparing it to the shift from assembly language to high-level programming languages. The authors also identify areas where serverless computing is pushing the boundaries, as well as the challenges and research opportunities that lie ahead. They predict that serverless computing will become a predominant model in the future of cloud computing.

\section{OpenWhisk}
OpenWhisk, an open-source serverless platform, has been the subject of numerous studies. Castro et al. \cite{castro2020openwhisk} provide an in-depth analysis of OpenWhisk's architecture and capabilities, offering valuable insights into its potential for hosting Swift-based serverless applications. This comprehensive study of OpenWhisk provides an essential foundation for understanding the possibilities and constraints of running Swift applications in a serverless environment.

\section{Serverless Computing and Languages}
The choice of programming language in serverless computing is an important aspect that has significant implications on the performance and efficiency of serverless applications. A study by \cite{implications-prog} investigates the impacts of programming language selection on the performance of serverless data processing pipelines. By implementing identical data processing pipelines in multiple languages (Java, Python, Go, and Node.js), the authors demonstrate that the performance of a pipeline can vary significantly based on the chosen language. Their study concludes that there is no one-size-fits-all language for serverless computing. Instead, they suggest that the fastest and most efficient pipeline could be achieved by adopting a hybrid approach, combining different languages to optimize various stages of the pipeline. This underlines the importance of careful language selection and performance profiling before deploying serverless applications, indicating that the choice of language can have substantial consequences on the overall performance and efficiency of serverless applications.

\section{Research Gaps}
Despite the wealth of knowledge available on serverless computing and OpenWhisk, there is a noticeable gap in the literature when it comes to using Swift as a serverless language. Few studies have specifically explored the performance, advantages, and challenges of using Swift in a serverless context, particularly on the OpenWhisk platform. This thesis aims to address this gap by providing an in-depth exploration of Swift as a serverless language on OpenWhisk. The goal is to gain a deeper understanding of the potential of Swift in this context and to contribute to the broader discourse on serverless computing and language choice.

